\section{Spline Quantile Function RNNs}
\label{sec:sqf-rnn}

\Textcite{Gasthaus2019} proposed a method for probabilistic forecasting by modeling 
the quantile function with monotonic regression splines. 
The proposed \gls{sqfrnn} model combines the ability to forecast time series 
from recurrent neural networks which was already done by the DeepAR model from \Textcite{Salinas2017}
with the flexibility of being able to 
model the quantile functions with linear splines. 

This flexibility and the fact that DeepAR uses autoregressive input with an RNN structure 
motivates the usage of this model instead of other models that don't 
have autoregressive input or an RNN structure like QRF or NNQF.

Let \(x_1, \ldots, x_n \in \R^D\) be the predictor values and 
\(z_1, \ldots, z_n \in \R\) be the target time series. Also, let \(\Theta\) 
be the model parameters, \(\boldsymbol{h}_t\) the network output of 
time step \(t\) and \(\theta_t\) the parameters of the conditional distribution \(\P(z_t | \theta_t)\).
The model works as follows:
Compute the network output \(\boldsymbol{h}_t = h(\boldsymbol{h}_{t-1}, x_t, z_{t-1}, \Theta)\) 
as well as the parameters \(\theta_t = \theta(\boldsymbol{h}_t, \Theta)\) for the distribution
\(\P(z_t | \theta_t)\). Here, the function \(h(\cdot)\) is a multi-layer RNN with 
LSTM cells and \(\theta(\cdot)\) is a projection layer. 
The quantiles are then used to calculate the loss and train the model parameters \(\Theta\).
The process is illustrated in Figure \ref{fig:deepar-training}. 

\begin{figure}[h]%
    \centering
    \begin{tikzpicture}[yscale=-1,node distance=-\pgflinewidth]
    \tikzset{ReceptorNode/.style={circle, draw=black, fill=lightblue, thick, inner sep=2pt, minimum size=30pt}}
    \tikzset{Placeholder/.style={circle, thick, inner sep=2pt, minimum size=30pt}}
    \tikzset{Connection/.style={->, line width=0.5mm}}
    \newcommand{\mynode}[3]{
        \node[ReceptorNode] (circ-#2) at (#1, 0) {\(\boldsymbol{h}_{#2}\)};
        \node (x-#2) at (#1, 1.5) {\(x_{#2}, y_{#3}\)};
        \node (y-#2) at (#1, -1.5) {\(\P(y_{#2}|\boldsymbol{h}_{#2})\)};

        \draw[Connection] (circ-#2) -- (y-#2);
        \draw[Connection] (x-#2)    -- (circ-#2);
    }
    \newcommand{\placeholder}[2]{
        \node[Placeholder] (circ-#2) at (#1, 0) {\(\cdots\)};
        \node (x-#2) at (#1, 1.5) {};
        \node (y-#2) at (#1, -1.5) {\phantom{\(\P(y_{#2}|\boldsymbol{h}_{#2})\)}};
    }
    \newcommand{\connect}[2]{
        \draw[Connection] (circ-#1) -- (circ-#2);
    }

    % Create nodes
    \mynode{1 * 2.5}{1}{0}
    \onslide<2->{
        \mynode{2 * 2.5}{2}{1}
        \connect{1}{2}    
    }
    \onslide<2->{
        \mynode{3 * 2.5}{3}{2}
        \connect{2}{3}    
    }
    \onslide<3->{
        \placeholder{4 * 2.5}{4}
        \connect{3}{4}
    }
    % Last node is called "n"
    \onslide<4->{
        \mynode{5*2.5}{n}{n-1}
        \connect{4}{n}
    }
\end{tikzpicture}
    \caption[DeepAR Training]{DeepAR Training. 
    The state of the network \(\boldsymbol{h}_t = h(\boldsymbol{h}_{t-1}, x_t, z_{t-1}, \Theta)\) 
    is calculated from the previous state \(\boldsymbol{h}_{t-1}\), 
    the predictor value \(x_t\) and the previous observed target point \(z_{t-1}\). 
    The target distribution \(\P(z_t | \boldsymbol{h}_t)\) 
    is calculated from the state of the network \(\boldsymbol{h}_t\).}%
    \label{fig:deepar-training}%
\end{figure}

For the prediction step, the target time series are not known. 
The known history of the time series \(z_1, \ldots, z_{t_0}\) is fed into the 
model and for \(t > t_0\), samples \(\tilde{z}_t \sim \P(z_t | \theta_t)\) 
are generated and fed back into the model for the next time step.
The process is illustrated in Figure \ref{fig:deepar-predicting}.

\begin{figure}[h]%
    \centering
    \begin{tikzpicture}[yscale=-1,node distance=-\pgflinewidth]
    \tikzset{ReceptorNode/.style={circle, draw=black, fill=lightblue, thick, inner sep=2pt, minimum size=30pt}}
    \tikzset{Placeholder/.style={circle, thick, inner sep=2pt, minimum size=30pt}}
    \tikzset{Connection/.style={->, line width=0.5mm}}
    \newcommand{\mynode}[3]{
        \node[ReceptorNode] (circ-#2) at (#1, 0) {\(\boldsymbol{h}_{#2}\)};
        \node (x-#2) at (#1, 1.5) {\(x_{#2}, z_{#3}\)};
        \node (y-#2) at (#1, -1.5) {\(\P(z_{#2}|\boldsymbol{h}_{#2})\)};

        \draw[Connection] (circ-#2) -- (y-#2);
        \draw[Connection] (x-#2)    -- (circ-#2);
    }
    \newcommand{\mynodewithresult}[3]{
        \node[ReceptorNode] (circ-#2) at (#1, 0) {\(\boldsymbol{h}_{#2}\)};
        \node (x-#2) at (#1, 1.5) {\(x_{#2}, z_{#3}\)};
        \node (y-#2) at (#1, -1.5) {\(\P(z_{#2}|\boldsymbol{h}_{#2})\)};
        \node (z-#2) at (#1, -2.5) {\(\tilde{z}_{#2}\)};

        \draw[Connection] (circ-#2) -- (y-#2);
        \draw[Connection] (x-#2)    -- (circ-#2);
        \draw[Connection] (y-#2)    -- (z-#2);
    }
    \newcommand{\mynodewithresultinputsampled}[3]{
        \node[ReceptorNode] (circ-#2) at (#1, 0) {\(\boldsymbol{h}_{#2}\)};
        \node (x-#2) at (#1, 1.5) {\(x_{#2}, \tilde{z}_{#3}\)};
        \node (y-#2) at (#1, -1.5) {\(\P(z_{#2}|\boldsymbol{h}_{#2})\)};
        \node (z-#2) at (#1, -2.5) {\(\tilde{z}_{#2}\)};

        \draw[Connection] (circ-#2) -- (y-#2);
        \draw[Connection] (x-#2)    -- (circ-#2);
        \draw[Connection] (y-#2)    -- (z-#2);
    }
    \newcommand{\placeholder}[2]{
        \node[Placeholder] (circ-#2) at (#1, 0) {\(\cdots\)};
        \node (x-#2) at (#1, 1.5) {};
        \node[opacity=0] (y-#2) at (#1, -1.5) {\(\P(z_{#2}|h_{#2})\)};
    }
    \newcommand{\connect}[2]{
        \draw[Connection] (y-#1)    -- (circ-#2);
        \draw[Connection] (circ-#1) -- (circ-#2);
    }

    % Create nodes
    \mynode{1 * 2.5}{T}{T-1}
    \draw[Connection] ([xshift=-0.5cm]circ-T.west) -- (circ-T);
    \onslide<2->{
        \mynodewithresult{2 * 2.5}{T+1}{T}
        \connect{T}{T+1}
    }
    \onslide<3->{
        \mynodewithresultinputsampled{3 * 2.5}{T+2}{T+1}
        \connect{T+1}{T+2}
    }
    \onslide<4->{
        \placeholder{4 * 2.5}{T+3}
        \connect{T+2}{T+3}
    }
    % Last node is called "T+n"
    \onslide<5->{
        \mynodewithresultinputsampled{5*2.5}{T+n}{T+n-1}
        \connect{T+3}{T+n}
    }
\end{tikzpicture}
    \caption[DeepAR Predicting]{DeepAR Predicting. 
    In the prediction steps, the target data \(z_{t-1}\) is not available. 
    To combat this, we sample \(\tilde{z}_{t-1} \sim \P(z_{t-1} | \boldsymbol{h}_{t-1})\) 
    from the previous step and use it as input instead of \(z_{t-1}\).}%
    \label{fig:deepar-predicting}%
\end{figure}

While the DeepAR model is trained by maximizing the likelihood function, 
the SQF-RNN model is trained by minimizing the CRPS (see \ref{ch:crps}) 
which can be computed effectively for spline-based quantile functions.
\todo{Maybe calculation of spline CRPS?}

A linear spline with \(L\) pieces is of the form 
\[ s(x; \gamma, b, d) = \gamma + \sum_{l=0}^L b_l (x - d_l)_+, 
\quad b, d \in \R^{L+1}. \]
Since we want a monotone spline, we need to create constraints for \(b_l\) and \(d_l\).
We want \(d_l < d_{l+1}\) for ordered knot positions. To achieve this 
in the neural network, we set \(d_0 = 0\) and \(d_l = \sum_{j=1}^l \delta_j\), 
where \(\delta_j \geq 0\) and \(\sum_{j=1}^L \delta_j = 1\) since the domain 
of the quantile function is \([0, 1]\). 
We also want monotonicity: the slope \(m_l\) between two knots is given by 
\(m_l = \sum_{j=0}^l b_j\). We need to make sure that \(m_l \geq 0 \forall l\).
If we set \(b_l = \beta_l - \beta_{l-1}\) and \(b_0 = \beta_0\) with \(\beta_l \geq 0 \forall l\), 
we get \(m_l = \sum_{j=0}^l b_j = \beta_l \geq 0\).
We can therefore model our spline with the parameter 
\(\theta = (\gamma, \beta, \delta)\), \(\gamma \in \R, \beta \in [0,\infty)^{L}, 
\delta \in \set{ \delta \in [0,1]^L: \sum_{j=1}^L \delta_j = 1 }\).