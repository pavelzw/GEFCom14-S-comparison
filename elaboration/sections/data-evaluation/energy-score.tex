\subsection{Energy Score}
\label{sec:energy-score-explanation}

The energy score is a proper scoring rule that is used for 
evaluating multivariate forecasts. A time series \((y_t)_t\) 
can be evaluated in this fashion by looking at a sequence of \(H\) 
points, i.e. \((y_t, \ldots, y_{t-H+1})_t \in \R^H\). 
The scoring rule is defined as follows:
\[ \mathrm{S}(P, y) = \mathbb{E}_P ||Y-y||_2 - \frac{1}{2} \mathbb{E}_P ||Y-Y'||_2, \]
where \(Y'\) is an i.i.d. copy of \(Y \in \R^H\), so it is drawn independently with the same distribution \(P\) 
which is an \(H\)-dimensional probability distribution.

This scoring rule takes the time series attributes, namely the 
temporal dependencies of adjacent data points, into account while 
the pinball loss only evaluates the score pointwise.