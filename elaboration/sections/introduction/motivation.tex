\section{Motivation}
\label{sec:motivation}

% paragraph 1: 
% why is energy forecasting important (with citations)
% highlight solar forecasting as being of particular interest & 1-2 challenges

Energy forecasting has become increasingly important in the recent years 
because of a higher energy consumption and a higher demand 
for renewable energies. 
\Textcite{Hong2016} state that energy supply, demand and prices are becoming increasingly 
volatile and unpredictable. 
Solar power forecasting in particular is an interesting field since it can be modeled through predictor variables 
like surface solar radiation, total cloud cover and precipitation. 
As material cost becomes lower and efficiency becomes higher, solar energy generation is becoming more and more important. 
\Textcite{Hong2016} describe the following common challenges for solar power forecasting.
Data cleansing is important since the real-world data for solar forecasting is not always clean. 
Different probabilistic forecasting methodologies can be used. One challenge is to find the most effective method at generating probabilistic forecasts. 
Combining multiple forecasts into one forecast often leads to a more robust point forecasts -- how can we do that for probabilistic ones?

% paragraph 2: 
% motivate the importance of probabilistic solar forecasting
% solar forecasting is immature in comparison to other types

Because modeling the uncertainty of a forecast is relevant for planning the energy household of the electric grid, 
probabilistic forecasts make sense instead of only point forecast. 
\Textcite{Hong2016} state that wind power forecasting is 
more mature than solar energy forecasting. This is because 
wind power forecasing is the closest to meteorological forecasting, 
where probabilistic forecasting is well-established and commonly accepted.
Although solar power forecasting has the same subdomain as wind forecasting, 
namely renewable energy generation, it is neither in the 
point forecasting nor in the probabilistic forecasting domain as mature 
as wind power forecasting. 

% paragraph 3: 
% motivate deep learning for probabilistic solar forecasting
% motivate nonparametric deep learning for probabilistic solar forecasting
% highlight how there is no comparison yet

Because of the large amount of data that is being used for solar power forecasting, it makes sense to look at 
deep learning models fore probabilistic solar forecasting. 
Deep learning often provides good results for complex problems with a lot of data to process. 
There are two types of forecasts: parametric ones and nonparametric ones. 
Parametric learning often yields good results if the underlying probability distribution class is known, e.g. \(\mathcal{N}(\mu, \sigma^2)\). 
Then, the parameters of the distribution are estimated. 
Nonparametric learning doesn't make these assumptions. 
The advantage here is that this type of learning is applicable to problems where the underlying 
probability distribution class is unknown -- like in solar energy forecasting. 


% paragraph 4:
% explain what you do in this thesis
% introduce models compared
% introduce data set

This is the reason why we will be comparing three different nonparametric machine learning models to tackle the probabilistic solar energy forecasting problem. 
The three models we take a look at are 
\begin{enumerate}
    \item Quantile Regression Forests proposed by \Textcite{Meinshausen2006},
    \item Nearest Neighbor Quantile Filters proposed by \Textcite{Ordiano2019} and
    \item Spline Quantile Functions RNN proposed by \Textcite{Gasthaus2019}.
\end{enumerate}
All of these models are so called distributional regression methods, meaning 
they predict a probabilistic distribution and try to minimize a loss function 
given the distribution and the target data.
We evaluate the models on the GEFCom2014 dataset and compare how they perform in different metrics like the pinball loss, the energy score and PIT histograms. 