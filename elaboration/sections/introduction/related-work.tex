\section{Related Work}
\label{sec:related-work}

This section covers related work concerning solar energy forecasting as well as 
the different models that we investigate. 
To get a broad overview of energy forecasting, 
\Textcite{Hong2020} offer a review of the entire field. 
They summarize research trends and offer an outlook into the future of energy forecasting. 
In 2014, \Textcite{Hong2016} hosted the Global Energy Forecasting Competition where 
solutions for the solar energy, wind energy, electric load and electric price forecasting 
problems were submitted. 
In this thesis, we evaluate different models on the solar energy problem 
of the previously mentioned competition. We will compare the following four models 
and discuss their suitability for solar power generation forecasting. 
The Quantile Random Forests method is proposed by \Textcite{Meinshausen2006}. 
In this work, its function is described 
and numerical examples that underline its predictive performance are given 
as well as proofs for some of the model's properties like consistency. 
Nearest Neighbors Quantile Filters are introduced by \Textcite{Ordiano2019}. 
They explain the advantages and test the model's performance on the GEFCom2014 dataset. 
We will also try out the model on the GEFCom2014 dataset and evaluate 
it in multiple kinds of manners.
The DeepAR model is proposed by \Textcite{Salinas2017}. 
It is a model for probabilistic forecasting using autoregressive recurrent neural networks. 
They evaluate the model on several 
real-world data sets and show that this model has accuracy improvements of around 
\(15\%\) compared to other state-of-the-art methods. 
\Textcite{Gasthaus2019} propose the SQF-RNN model, an extension of the DeepAR model 
that is supposed to be a more flexible method for probabilistic forecasting 
by modeling the quantile function with splines instead of a fixed distribution class.
They empirically demonstrate its effectiveness by evaluating the model on several real-world 
data sets and comparing it with the DeepAR model.