\chapter{Model description}
\label{ch:model-description}

In this chapter, we motivate why the models were selected and 
describe their basic functionality as well as their advantages.

Quantile Regression Forests have proven to be a very competitive 
and powerful quantile regression method for high-dimensional datasets. 
Quantile Regression Forests were used by three of the top five contestants 
in the solar energy track in the GEFCom2014 challenge (cf. \Textcite{Hong2016}, Table 11). 
There are also multiple libraries that implement Quantile Regression Forests.
Therefore, they do not only perform 
well but they are also simple to implement as a benchmark.
This is the reason why we use it as a competitive benchmark model.
Nearest Neighbor Quantile Regression is the method utilized by the first place 
in the GEFCom14 challenge. \Textcite{Ordiano2019} claim that NNQF uses a similar strategy but requires much less 
computation power in order to predict its output. 
\Textcite{Gasthaus2019} explain that SQF-RNN uses autoregressive input with an RNN structure and provides flexible 
conditional output distributions through spline quantile functions. 
They also show that it works well on large solar energy 
forecasting datasets.

In order to describe the model in the following subsections, we will introduce the necessary notation here:
\begin{itemize}
    \item \(n\) is the training data length
    \item \(x_1, \ldots, x_n \in \R^D\) are the predictor values of the time series
    \item \(y_1, \ldots, y_n \in \R\) are the corresponding target values
\end{itemize}

\section{Quantile Regression Forests}
\label{sec:qrf}

Quantile Regression Forests were first proposed by \Textcite{Meinshausen2006}
and have since then proven to be a powerful method for high-dimensional quantile 
regression and time series forecasting. 
The method works in a similar way as Random Forests which are introduced by \Textcite{Breiman2001}
with the main difference 
being that the predictor is not the mean over the trees but takes the form of different quantiles 
from the trees.

The performance of this algorithm is very competitive in comparison with other 
linear and tree-based methods.
Therefore, they provide a competitive baseline for this thesis. 

\Textcite{Meinshausen2006} also shows that Quantile Regression Forests are consistent 
under some specific assumptions about the distribution of the covariates, the proportion of observations, the splitting criterion and the 
continuity and monotonicity of the \gls{cdf}.

Quantile Regression Forests work as follows (cf. \Textcite{Meinshausen2006}):
\begin{enumerate}
    \item Grow \(k\) trees \(\func{T_1, \ldots, T_k}{\R^D}{\R}\) like in a Random Forest from the training data.
    \item For a given \(x\in\R^D\), calculate the distributional outputs \( \tilde{y}_i = T_i(x) \) for all \(i \in \set{1, \ldots, k}\), 
    where \(T_i(x)\) is the distributional output of the \(i\)-th tree when \(x\) is used as input.
    \item Calculate the empirical quantiles \(y_{(0.01)}, \ldots, y_{(0.99)}\) from the 
    combined forecast distribution, which can be obtained by averaging over all distributions 
    \(\tilde{y}_1, \ldots, \tilde{y}_k\).
\end{enumerate}

\section{Nearest Neighbor Quantile Filters}
\label{sec:nnqf}

Ordiano et al. (2020) proposed a method for probabilistic 
energy forecasting using quantile regression based on a Nearest Neighbor 
Quantile Filter. 
The method works as follows: first, the training set is modified 
by using the Nearest Neigbor Quantile Filters so that 
the training data directly represents a probabilistic distribution. 
Now, a regression model like an artificial neural network can 
train on this modified data set and learn the quantile function.

Let \(x_n \in \R^D\), \(1 \leq n \leq N\), be the \(n\)-th predictor 
and \(y_n \in \R\) the \(n\)-th value of the time series. 
The NNQF first searches the \(k\) nearest neighbors of \(x_n\). 
Let \(J \subset \set{1, \ldots, N}\) be the indices of 
those nearest neighbors. 
The probabilistic distribution of \(y_n\) can be approximated 
by calculating the empirical quantile \(\tilde{y}_{(q),n}\) of 
\(\set{y_n \;|\; n\in J}\) for each \(q \in \set{0.01, \ldots, 0.99}\). 

After repeating this procedure for each entry in the time series, 
we get a vector 
\[ \tilde{y}_{(q)} = \begin{pmatrix}
    \tilde{y}_{(q), 1} \\ 
    \vdots \\
    \tilde{y}_{(q), N}
\end{pmatrix} \]
that form the modified training set combined with the 
predictors \(X = x_1, \ldots, x_N\).

With the modified training set, one can now train the regression model 
for each quantile. Common examples are polynomial regression or 
artificial neural networks. 

This method has three main advantages: 
\begin{enumerate}
    \item \(q\) is a free parameter and can be changed to any \(q \in (0,1)\),
    \item the technique for the quantile regression is not specified, 
    any technique can be used,
    \item the calculation of the nearest neighbors and the modified 
    training set only needs to be done once, you can save time when 
    using multiple quantile regressions. Also, the original dataset 
    does not need to be saved after that.
\end{enumerate}

In comparison to most other \(k\)-Nearest Neighbors quantile 
regression techniques, the nearest neighbors are only calculated once 
and then the regression model is trained on the modified training data. 
A regular \(k\)-Nearest Neighbor Quantile Regression algorithm 
calculates the nearest neighbors every time when a forecast is conducted 
which is computationally more expensive in the long run.

\section{Spline Quantile Function RNNs}
\label{sec:sqf-rnn}

\Textcite{Gasthaus2019} proposed a method for probabilistic forecasting by modeling 
the quantile function with monotonic regression splines. 
The proposed \gls{sqfrnn} model combines the ability to forecast time series 
from recurrent neural networks which was already done by the DeepAR model from \Textcite{Salinas2017}
with the flexibility of being able to 
model the quantile functions with linear splines. 

Let \(x_1, \ldots, x_n \in \R^D\) be the predictor values and 
\(z_1, \ldots, z_n \in \R\) be the target time series. Also, let \(\Theta\) 
be the model parameters, \(\boldsymbol{h}_t\) the network output of 
time step \(t\) and \(\theta_t\) the parameters of the conditional distribution \(\P(z_t | \theta_t)\).
The model works as follows:
Compute the network output \(\boldsymbol{h}_t = h(\boldsymbol{h}_{t-1}, z_{t-1}, x_t, \Theta)\) 
as well as the parameters \(\theta_t = \theta(\boldsymbol{h}_t, \Theta)\) for the distribution
\(\P(z_t | \theta_t)\). The process is illustrated in Figure \ref{fig:deepar-training}.

\begin{figure}[h]%
    \centering
    \begin{tikzpicture}[yscale=-1,node distance=-\pgflinewidth]
    \tikzset{ReceptorNode/.style={circle, draw=black, fill=lightblue, thick, inner sep=2pt, minimum size=30pt}}
    \tikzset{Placeholder/.style={circle, thick, inner sep=2pt, minimum size=30pt}}
    \tikzset{Connection/.style={->, line width=0.5mm}}
    \newcommand{\mynode}[3]{
        \node[ReceptorNode] (circ-#2) at (#1, 0) {\(\boldsymbol{h}_{#2}\)};
        \node (x-#2) at (#1, 1.5) {\(x_{#2}, z_{#3}\)};
        \node (y-#2) at (#1, -1.5) {\(\P(z_{#2}|\boldsymbol{h}_{#2})\)};

        \draw[Connection] (circ-#2) -- (y-#2);
        \draw[Connection] (x-#2)    -- (circ-#2);
    }
    \newcommand{\placeholder}[2]{
        \node[Placeholder] (circ-#2) at (#1, 0) {\(\cdots\)};
        \node (x-#2) at (#1, 1.5) {};
        \node[opacity=0] (y-#2) at (#1, -1.5) {\(\P(z_{#2}|\boldsymbol{h}_{#2})\)};
    }
    \newcommand{\connect}[2]{
        \draw[Connection] (y-#1)    -- (circ-#2);
        \draw[Connection] (circ-#1) -- (circ-#2);
    }

    % Create nodes
    \mynode{1 * 2.5}{1}{0}
    \mynode{2 * 2.5}{2}{1}
    \connect{1}{2}
    \mynode{3 * 2.5}{3}{2}
    \connect{2}{3}
    \placeholder{4 * 2.5}{4}
    \connect{3}{4}
    % Last node is called "n"
    \mynode{5*2.5}{n}{n-1}
    \connect{4}{n}
\end{tikzpicture}
    \caption{DeepAR Training}%
    \label{fig:deepar-training}%
\end{figure}

For the prediction step, the target time series are not known. 
The known history of the time series \(z_1, \ldots, z_{t_0}\) is fed into the 
model and for \(t > t_0\), samples \(\tilde{z}_t \sim \P(z_t | \theta_t)\) 
are generated and fed back into the model for the next time step.
The process is illustrated in Figure \ref{fig:deepar-predicting}.

\begin{figure}[h]%
    \centering
    \begin{tikzpicture}[yscale=-1,node distance=-\pgflinewidth]
    \tikzset{ReceptorNode/.style={circle, draw=black, fill=lightblue, thick, inner sep=2pt, minimum size=30pt}}
    \tikzset{Placeholder/.style={circle, thick, inner sep=2pt, minimum size=30pt}}
    \tikzset{Connection/.style={->, line width=0.5mm}}
    \tikzset{LightConnection/.style={->, dashed, line width=0.3mm, opacity=0.25}}
    \newcommand{\mynode}[3]{
        \node[ReceptorNode] (circ-#2) at (#1, 0) {\(\boldsymbol{h}_{#2}\)};
        \node (x-#2) at (#1, 1.5) {\(x_{#2}, z_{#3}\)};
        \node (y-#2) at (#1, -1.5) {\(\P(z_{#2}|\boldsymbol{h}_{#2})\)};

        \draw[Connection] (circ-#2) -- (y-#2);
        \draw[Connection] (x-#2)    -- (circ-#2);
    }
    \newcommand{\mynodewithresult}[3]{
        \node[ReceptorNode] (circ-#2) at (#1, 0) {\(\boldsymbol{h}_{#2}\)};
        \node (x-#2) at (#1, 1.5) {\(x_{#2}, z_{#3}\)};
        \node (y-#2) at (#1, -1.5) {\(\P(z_{#2}|\boldsymbol{h}_{#2})\)};
        \node (z-#2) at (#1, -2.5) {\(\tilde{z}_{#2}\)};

        \draw[Connection] (circ-#2) -- (y-#2);
        \draw[Connection] (x-#2)    -- (circ-#2);
        \draw[Connection] (y-#2)    -- (z-#2);
    }
    \newcommand{\mynodewithresultinputsampled}[3]{
        \node[ReceptorNode] (circ-#2) at (#1, 0) {\(\boldsymbol{h}_{#2}\)};
        \node (x-#2) at (#1, 1.5) {\(x_{#2}, \tilde{z}_{#3}\)};
        \node (y-#2) at (#1, -1.5) {\(\P(z_{#2}|\boldsymbol{h}_{#2})\)};
        \node (z-#2) at (#1, -2.5) {\(\tilde{z}_{#2}\)};

        \draw[Connection] (circ-#2) -- (y-#2);
        \draw[Connection] (x-#2)    -- (circ-#2);
        \draw[Connection] (y-#2)    -- (z-#2);
    }
    \newcommand{\placeholder}[2]{
        \node[Placeholder] (circ-#2) at (#1, 0) {\(\cdots\)};
        \node (x-#2) at (#1, 1.5) {\phantom{\(x_{#2}, \tilde{z}_{#2}\)}};
        \node (y-#2) at (#1, -1.5) {\phantom{\(\P(z_{#2}|h_{#2})\)}};
        \node (z-#2) at (#1, -2.5) {\phantom{\(\tilde{z}_{#2}\)}};
    }
    \newcommand{\connect}[2]{
        \draw[Connection] (circ-#1) -- (circ-#2);
    }
    \newcommand{\connectsampled}[2]{
        \draw[LightConnection] (z-#1) -- (x-#2);
        \draw[Connection] (circ-#1)   -- (circ-#2);
    }

    % Create nodes
    \mynode{1 * 2.5}{t_0}{t_0-1}
    \draw[Connection] ([xshift=-0.5cm]circ-t_0.west) -- (circ-t_0);
    \onslide<2->{
        \mynodewithresult{2 * 2.5}{t_0+1}{t_0}
        \connect{t_0}{t_0+1}
    }
    \onslide<3->{
        \mynodewithresultinputsampled{3 * 2.5}{t_0+2}{t_0+1}
        \connectsampled{t_0+1}{t_0+2}
    }
    \onslide<4->{
        \placeholder{4 * 2.5}{t_0+3}
        \connectsampled{t_0+2}{t_0+3}
    }
    % Last node is called "t_0+T"
    \onslide<5->{
        \mynodewithresultinputsampled{5*2.5}{t_0+T}{t_0+T-1}
        \connectsampled{t_0+3}{t_0+T}
    }
\end{tikzpicture}
    \caption{DeepAR Predicting}%
    \label{fig:deepar-predicting}%
\end{figure}

While the DeepAR model is trained by maximizing the likelihood function, 
the SQF-RNN model is trained by minimizing the CRPS (see \ref{ch:crps}) 
which can be computed effectively for spline-based quantile functions.

A linear spline with \(L\) pieces is of the form 
\[ s(x; \gamma, b, d) = \gamma + \sum_{l=0}^L b_l (x - d_l)_+, 
\quad b, d \in \R^{L+1}. \]
Since we want a monotone spline, we need to create constraints for \(b_l\) and \(d_l\).
We want \(d_l < d_{l+1}\) for ordered knot positions. To achieve this 
in the neural network, we set \(d_0 = 0\) and \(d_l = \sum_{j=1}^l \delta_j\), 
where \(\delta_j \geq 0\) and \(\sum_{j=1}^L \delta_j = 1\) since the domain 
of the quantile function is \([0, 1]\). 
We also want monotonicity: the slope \(m_l\) between two knots is given by 
\(m_l = \sum_{j=0}^l b_j\). We need to make sure that \(m_l \geq 0 \forall l\).
If we set \(b_l = \beta_l - \beta_{l-1}\) and \(b_0 = \beta_0\) with \(\beta_l \geq 0 \forall l\), 
we get \(m_l = \sum_{j=0}^l b_j = \beta_l \geq 0\).
We can therefore model our spline with the parameter 
\(\theta = (\gamma, \beta, \delta)\), \(\gamma \in \R, \beta \in [0,\infty)^{L}, 
\delta \in \set{ \delta \in [0,1]^L: \sum_{j=1}^L \delta_j = 1 }\).

Let \(z_{i,t} \in \R\) be the value of the \(i\)-th time series at time \(t\) and 
\(x_{i,t} \in \R^D\) be the corresponding predictors for the time series. 
The predictors are assumed to be available during prediction. 

The setup of the model is the following: 
During training, the model takes \(x_{i,t}\) as well as \(z_{i, t-1}\) 
and \(h_{i, t-1}\) -- the previous state of the network -- 
as input and calculates \(h_{i,t} = r(h_{i, t-1}, z_{i, t-1}, x_{i, t})\)
which is then used to calculate the quantile outputs 
\(\theta_{i,t} = \theta(h_{i,t}, \phi)\).
\(r(\cdot)\) is a multi-layer RNN with LSTM cells and \(\theta(\cdot)\) 
is a projection layer.
The quantiles are then used to calculate the CRPS loss and train the model.