\section{Energy Score}

The energy score is a proper scoring rule that is used for 
evaluating multi dimensional forecasts. A time series \((y_t)_t\) 
can be evaluated in this fashion by looking at a sequence of \(H\) 
points, i.e. \((y_t, \ldots, y_{t-H+1})_t \in \R^H\). 
The scoring rule is defined as follows:
\[ \mathrm{S}(P, y) = \mathbb{E}_P ||Y-y||_2 - \frac{1}{2} \mathbb{E}_P ||Y-Y'||_2, \]
where \(Y'\) is an i.i.d. copy of \(Y\), so it is drawn independently with the same distribution \(P\).

This scoring rule takes the time series attributes, namely the 
correlation of adjacent data points, into account while 
the pinball loss only evaluates the score pointwise.

The results of the energy score are shown in Figure \ref{fig:energy-score}. 
We can observe that the SQF-RNN model performs better than or at least similar 
to the NNQF and QRF models in every month with July and August as exception. 
Another thing to note is that the DeepAR model always performs worse than the SQF-RNN model. 

\begin{figure}[ht]
    \centering
    \begin{tikzpicture}[scale=0.6]
    \pgfplotsset{every axis/.style={mlineplot}}
    \begin{axis}[title=Energy score, 
                 xlabel=Month, 
                 ylabel=loss, 
                 xtick={4,6,8,10,12,14}, 
                 xticklabels={Jul, Sep, Nov, Jan, Mar, May}]
        % NNQF
        \addplot coordinates {(4, 0.30482) (5, 0.40201) (6, 0.37712) (7, 0.42732) (8, 0.42651) (9, 0.41919) (10, 0.41308) (11, 0.37499) (12, 0.36515) (13, 0.31821) (14, 0.28263) (15, 0.30319)};
        \addlegendentry{NNQF}
        % QRF
        \addplot coordinates {(4, 0.30431) (5, 0.40166) (6, 0.37702) (7, 0.42787) (8, 0.42602) (9, 0.41971) (10, 0.41166) (11, 0.37581) (12, 0.36495) (13, 0.31922) (14, 0.28258) (15, 0.30330)};
        \addlegendentry{QRF}
        % SQF-RNN
        \addplot coordinates {(4, 0.42407) (5, 0.54152) (6, 0.37154) (7, 0.30926) (8, 0.32410) (9, 0.33759) (10, 0.36245) (11, 0.31466) (12, 0.39435) (13, 0.34716) (14, 0.29842) (15, 0.35548)};
        \addlegendentry{SQF-RNN}
        % DeepAR
        \addplot coordinates {(4, 0.50212) (5, 0.64805) (6, 0.48501) (7, 0.34316) (8, 0.45956) (9, 0.39791) (10, 0.39915) (11, 0.46347) (12, 0.45471) (13, 0.42150) (14, 0.35721) (15, 0.37121)};
        \addlegendentry{DeepAR}
    \end{axis}
\end{tikzpicture}
    \caption[Energy score]{Energy score. 
    This graph plots the scores of the models for each month of the dataset competition. A lower score means a better performance.}
    \label{fig:energy-score}
\end{figure}

QRF returns a mean score of \(0.36759\), NNQF a mean score of \(0.36736\), 
SQF-RNN a mean score of \(0.36644\) and DeepAR a mean score of \(0.44235\).