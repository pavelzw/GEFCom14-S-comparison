\pdfbookmark[1]{Abstract}{Abstract}

\chapter*{Abstract}

\begin{center}
  \begin{minipage}{12cm}
    \begin{sloppypar}
      As the share of electricity from regenerative sources is growing constantly, 
      it becomes increasingly important to predict the power generation of these sources. 
      In order to model the uncertainty of a forecast, one often predicts the probability 
      distribution of the corresponding random variable. There are many different nonparametric machine learning approaches 
      to the solar power generation forecasting problem. Hence, this thesis compares three approaches 
      in order to predict solar power generation, namely Quantile Random Forests, Nearest Neighbor Quantile Filters and 
      Spline Quantile Functions RNN. The performance of the models is evaluated on the Global Energy Forecasting Competition 2014 
      dataset. 
      We find that Quantile Regression Forests manage to get the best results following 
      the Nearest Neighbor Quantile Filters model and the Spline Quantile Functions RNN model as third place of those three models. 
    \end{sloppypar}
  \end{minipage}
\end{center}