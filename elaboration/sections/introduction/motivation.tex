\section{Motivation}
\label{sec:motivation}

% paragraph 1: 
% why is energy forecasting important (with citations)
% highlight solar forecasting as being of particular interest & 1-2 challenges

In the last decade, climate change has become an increasingly urgent problem. 
In order to reduce the amount of CO2 emitted due to energy production, 
the transition towards renewable energy sources is important. 
The photovoltaic share in the energy mix has seen continuous growth in 
the past years (\citeauthor{Snapshot2016}, \citedate{Snapshot2016}). 

Solar energy production fluctuates strongly as variables such as the weather change. 
\Textcite{Meer2018} argue that accurate forecasting of power generation is needed to properly 
integrate the renewable energy sources into the energy generation mix. 
In terms of forecasting methods, solar power is 
an interesting problem since it can be modeled through predictor variables 
like surface solar radiation, total cloud cover and precipitation 
for which high-quality forecasts from numerical weather prediction models are readily available. 

% paragraph 2: 
% motivate the importance of probabilistic solar forecasting
% solar forecasting is immature in comparison to other types

Energy forecasting has traditionally focused on point forecasts 
but nowadays, there is a substantial interest in probabilistic forecasts.
Because modeling the uncertainty of a forecast is relevant for planning the energy household of the electric grid, 
probabilistic forecasts make sense instead of only point forecasts (\citeauthor{Meer2018}, \citedate{Meer2018}). 
One challenge is to find the most effective method from all different probabilistic forecasting methodologies. 
\Textcite{Hong2016} state that wind power forecasting is 
more mature than solar energy forecasting.
This is most likely due to 
wind power forecasting being the closest to meteorological forecasting, 
where probabilistic forecasting is well-established and commonly accepted. 

% paragraph 3: 
% motivate deep learning for probabilistic solar forecasting
% motivate nonparametric deep learning for probabilistic solar forecasting
% highlight how there is no comparison yet

Because of the large amount of data that is being used for solar power forecasting, it makes sense to look at 
deep learning models for probabilistic solar forecasting. 
Deep learning often provides good results for complex problems with a lot of data to process. 
Generally, probabilistic forecasts can be distinguished into two types of forecasts: parametric, and nonparametric ones. 
Parametric learning assumes an underlying probability distribution, e.g., \(\mathcal{N}(\mu, \sigma^2)\). Here, the parameters 
of the distribution are estimated. 
Nonparametric learning does not make these assumptions. 
The advantage here is that this type of learning is applicable to problems where the underlying 
probability distribution class is unknown or difficult to specify -- like in solar energy forecasting. 

% paragraph 4:
% explain what you do in this thesis
% introduce models compared
% introduce data set

For this reason, we compare three different nonparametric machine learning models 
that tackle the probabilistic solar energy forecasting problem. 
The three models we take a look at are 
\begin{enumerate}
    \item Quantile Regression Forests (QRF) proposed by \Textcite{Meinshausen2006},
    \item Nearest Neighbor Quantile Filters (NNQF) proposed by \Textcite{Ordiano2019} and
    \item Spline Quantile Functions Recurrent Neural Network (SQF-RNN) proposed by \Textcite{Gasthaus2019}.
\end{enumerate}
All of these models are so called distributional regression methods, meaning 
they predict a probability distribution and try to minimize a loss function 
given the distribution and the target data.
We evaluate the models on the Global Energy Forecasting Competition 2014 (GEFCom2014) dataset and compare how well they perform 
in different metrics like the pinball loss (cf. \ref{sec:pinball-loss-explanation}), the energy score and PIT histograms. 