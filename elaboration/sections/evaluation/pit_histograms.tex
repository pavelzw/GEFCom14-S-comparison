\section{PIT Histograms}

In this section, we will look at the PIT histograms of the different forecasting models. They are introduced in section \ref{sec:pit-histogram-explanation}.

\begin{figure}[h]%
    \centering
    \subfloat[\centering Quantile Random Forests]{{\includegraphics[width=0.45\textwidth]{plots/pit/pit_qrf.pdf} \label{fig:pit-qrf} }}
    \subfloat[\centering Nearest Neighbor Quantile Filters]{{\includegraphics[width=0.45\textwidth]{plots/pit/pit_nnqf.pdf} \label{fig:pit-nnqf} }} \\
    \subfloat[\centering DeepAR]{{\includegraphics[width=0.45\textwidth]{plots/pit/pit_deepar.pdf} \label{fig:pit-deepar} }}
    \subfloat[\centering Spline Quantile Functions RNN]{{\includegraphics[width=0.45\textwidth]{plots/pit/pit_sqf-rnn.pdf} \label{fig:pit-sqf-rnn} }}
    \caption[PIT histograms]{Probability integral transform of each model. 
    If the distribution of the PIT looks like \(U(0,1)\), it's probabilistically calibrated.}%
    \label{fig:pit}%
\end{figure}

Because the models act differently at each hour, it makes sense 
to look at the PIT histograms for each hour separately. 
The PIT histograms broken down into each hour are shown in Figure \ref{fig:pit-qrf-by-hour}, Figure \ref{fig:pit-nnqf-by-hour}, 
Figure \ref{fig:pit-deepar-by-hour} and Figure \ref{fig:pit-sqf-rnn-by-hour} 
while Figure \ref{fig:pit} shows the PIT histograms averaged over each hour.

One can observe that the PIT histogram of QRF (\ref{fig:pit-qrf}) and NNQF (\ref{fig:pit-nnqf}) are approximately 
\(U(0,1)\)-distributed but the forecasts during the night are overdispersed in the QRF case and 
skewed in the NNQF model.
The DeepAR and SQF-RNN model perform similarly: they are both skewed with a high density at \(0\). 
This means that the actual output lies often in the lower quantiles of the prediction. 