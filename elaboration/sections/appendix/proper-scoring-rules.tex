\section{Proper Scoring Rules}
\label{sec:proper-scoring-rules}

Proper scoring rules are discussed in detail by \Textcite{Gneiting2007} and \Textcite{Gneiting2014}.
To measure the error of a probabilistic forecast, one usually uses a 
proper scoring rule. 
Let \(\mathcal{P}\) be a convex class of probability measures on the 
measurable space \((\Omega, \mathcal{A})\).
A \textit{scoring rule} is any extended real-valued function 
\[ \func{S}{\mathcal{P} \times \Omega}{\overline{\R}} \]
such that \(S(P, \cdot)\) is \(\mathcal{P}\)-quasiintegrable for all 
\(P\in \mathcal{P}\).
A scoring rule \(S\) is \textit{proper} relative to a convex subclass 
\(\mathcal{P}_0 \subseteq \mathcal{P}\) if
\[ \int_\Omega S(Q, \omega) \mathrm{d}Q(\omega) \leq \int_\Omega S(P, \omega) \mathrm{d}Q(\omega) \]
for all \(P, Q \in \mathcal{P}_0\). It is \textit{strictly proper} 
if equality holds iff. \(P = Q\).

This property encourages honest and careful quotes by the forecaster. 