\section{Spline Quantile Function RNNs}
\label{sec:sqf-rnn}

Normal neural networks don't use the temporal correlation of a time series. 
RNNs on the other hand have a layer that uses the current output of the network for the 
next time step. With this, RNNs can use their internal state as memory and 
are therefore able to process the temporal correlation of the time series. 
That's why it makes sense to look at how RNNs perform on our dataset.

DeepAR is an autoregressive recurrent neural network proposed by 
\Textcite{Salinas2017} for the prediction of time series. It models 
a known distribution like the normal distribution or the Student's \(t\)-distribution. 

\Textcite{Gasthaus2019} builds on top of that: they proposed \gls{sqfrnn}, 
a model that works just like DeepAR but doesn't use a predetermined output 
distribution. Instead they model the conditional distribution 
by spline quantile functions. 

This flexibility and the fact that DeepAR uses autoregressive input with an RNN structure 
motivates the usage of this model instead of other models that don't 
have autoregressive input or an RNN structure like QRF or NNQF.

Let \(\Theta\) be the model parameters, \(\boldsymbol{h}_t\) the network output of 
time step \(t\) and \(\theta_t\) the parameters of the conditional distribution \(\P(y_t | \theta_t)\).
The model works as follows:
Compute the network output \(\boldsymbol{h}_t = h(\boldsymbol{h}_{t-1}, x_t, y_{t-1}, \Theta)\) 
as well as the parameters \(\theta_t = \theta(\boldsymbol{h}_t, \Theta)\) for the distribution
\(\P(y_t | \theta_t)\). 
Here, the function \(h(\cdot)\) is a multi-layer RNN with 
Long short-term memory (LSTM, a type of RNN) cells and \(\theta(\cdot)\) is a projection layer, 
i.e. a linear map from a high dimensional space into a lower dimesnional space. 
The quantiles are then used to calculate the loss and train the model parameters \(\Theta\).
The process is illustrated in Figure \ref{fig:deepar-training}. 

\begin{figure}[h]%
    \centering
    \begin{tikzpicture}[yscale=-1,node distance=-\pgflinewidth]
    \tikzset{ReceptorNode/.style={circle, draw=black, fill=lightblue, thick, inner sep=2pt, minimum size=30pt}}
    \tikzset{Placeholder/.style={circle, thick, inner sep=2pt, minimum size=30pt}}
    \tikzset{Connection/.style={->, line width=0.5mm}}
    \newcommand{\mynode}[3]{
        \node[ReceptorNode] (circ-#2) at (#1, 0) {\(\boldsymbol{h}_{#2}\)};
        \node (x-#2) at (#1, 1.5) {\(x_{#2}, z_{#3}\)};
        \node (y-#2) at (#1, -1.5) {\(\P(z_{#2}|\boldsymbol{h}_{#2})\)};

        \draw[Connection] (circ-#2) -- (y-#2);
        \draw[Connection] (x-#2)    -- (circ-#2);
    }
    \newcommand{\placeholder}[2]{
        \node[Placeholder] (circ-#2) at (#1, 0) {\(\cdots\)};
        \node (x-#2) at (#1, 1.5) {};
        \node[opacity=0] (y-#2) at (#1, -1.5) {\(\P(z_{#2}|\boldsymbol{h}_{#2})\)};
    }
    \newcommand{\connect}[2]{
        \draw[Connection] (y-#1)    -- (circ-#2);
        \draw[Connection] (circ-#1) -- (circ-#2);
    }

    % Create nodes
    \mynode{1 * 2.5}{1}{0}
    \mynode{2 * 2.5}{2}{1}
    \connect{1}{2}
    \mynode{3 * 2.5}{3}{2}
    \connect{2}{3}
    \placeholder{4 * 2.5}{4}
    \connect{3}{4}
    % Last node is called "n"
    \mynode{5*2.5}{n}{n-1}
    \connect{4}{n}
\end{tikzpicture}
    \caption[DeepAR Training]{DeepAR Training. 
    The state of the network \(\boldsymbol{h}_t = h(\boldsymbol{h}_{t-1}, x_t, y_{t-1}, \Theta)\) 
    is calculated from the previous state \(\boldsymbol{h}_{t-1}\), 
    the predictor value \(x_t\) and the previous observed target point \(y_{t-1}\). 
    The target distribution \(\P(y_t | \boldsymbol{h}_t)\) 
    is calculated from the state of the network \(\boldsymbol{h}_t\).}%
    \label{fig:deepar-training}%
\end{figure}

For the prediction step, the target time series are not known. 
The known history of the time series \(y_1, \ldots, y_{t_0}\) is fed into the 
model and for \(t > t_0\), samples \(\tilde{y}_t \sim \P(y_t | \theta_t)\) 
are generated and fed back into the model for the next time step.
The process is illustrated in Figure \ref{fig:deepar-predicting}.

\begin{figure}[h]%
    \centering
    \begin{tikzpicture}[yscale=-1,node distance=-\pgflinewidth]
    \tikzset{ReceptorNode/.style={circle, draw=black, fill=lightblue, thick, inner sep=2pt, minimum size=30pt}}
    \tikzset{Placeholder/.style={circle, thick, inner sep=2pt, minimum size=30pt}}
    \tikzset{Connection/.style={->, line width=0.5mm}}
    \tikzset{LightConnection/.style={->, dashed, line width=0.3mm, opacity=0.25}}
    \newcommand{\mynode}[3]{
        \node[ReceptorNode] (circ-#2) at (#1, 0) {\(\boldsymbol{h}_{#2}\)};
        \node (x-#2) at (#1, 1.5) {\(x_{#2}, z_{#3}\)};
        \node (y-#2) at (#1, -1.5) {\(\P(z_{#2}|\boldsymbol{h}_{#2})\)};

        \draw[Connection] (circ-#2) -- (y-#2);
        \draw[Connection] (x-#2)    -- (circ-#2);
    }
    \newcommand{\mynodewithresult}[3]{
        \node[ReceptorNode] (circ-#2) at (#1, 0) {\(\boldsymbol{h}_{#2}\)};
        \node (x-#2) at (#1, 1.5) {\(x_{#2}, z_{#3}\)};
        \node (y-#2) at (#1, -1.5) {\(\P(z_{#2}|\boldsymbol{h}_{#2})\)};
        \node (z-#2) at (#1, -2.5) {\(\tilde{z}_{#2}\)};

        \draw[Connection] (circ-#2) -- (y-#2);
        \draw[Connection] (x-#2)    -- (circ-#2);
        \draw[Connection] (y-#2)    -- (z-#2);
    }
    \newcommand{\mynodewithresultinputsampled}[3]{
        \node[ReceptorNode] (circ-#2) at (#1, 0) {\(\boldsymbol{h}_{#2}\)};
        \node (x-#2) at (#1, 1.5) {\(x_{#2}, \tilde{z}_{#3}\)};
        \node (y-#2) at (#1, -1.5) {\(\P(z_{#2}|\boldsymbol{h}_{#2})\)};
        \node (z-#2) at (#1, -2.5) {\(\tilde{z}_{#2}\)};

        \draw[Connection] (circ-#2) -- (y-#2);
        \draw[Connection] (x-#2)    -- (circ-#2);
        \draw[Connection] (y-#2)    -- (z-#2);
    }
    \newcommand{\placeholder}[2]{
        \node[Placeholder] (circ-#2) at (#1, 0) {\(\cdots\)};
        \node (x-#2) at (#1, 1.5) {\phantom{\(x_{#2}, \tilde{z}_{#2}\)}};
        \node (y-#2) at (#1, -1.5) {\phantom{\(\P(z_{#2}|h_{#2})\)}};
        \node (z-#2) at (#1, -2.5) {\phantom{\(\tilde{z}_{#2}\)}};
    }
    \newcommand{\connect}[2]{
        \draw[Connection] (circ-#1) -- (circ-#2);
    }
    \newcommand{\connectsampled}[2]{
        \draw[LightConnection] (z-#1) -- (x-#2);
        \draw[Connection] (circ-#1)   -- (circ-#2);
    }

    % Create nodes
    \mynode{1 * 2.5}{t_0}{t_0-1}
    \draw[Connection] ([xshift=-0.5cm]circ-t_0.west) -- (circ-t_0);
    \onslide<2->{
        \mynodewithresult{2 * 2.5}{t_0+1}{t_0}
        \connect{t_0}{t_0+1}
    }
    \onslide<3->{
        \mynodewithresultinputsampled{3 * 2.5}{t_0+2}{t_0+1}
        \connectsampled{t_0+1}{t_0+2}
    }
    \onslide<4->{
        \placeholder{4 * 2.5}{t_0+3}
        \connectsampled{t_0+2}{t_0+3}
    }
    % Last node is called "t_0+T"
    \onslide<5->{
        \mynodewithresultinputsampled{5*2.5}{t_0+T}{t_0+T-1}
        \connectsampled{t_0+3}{t_0+T}
    }
\end{tikzpicture}
    \caption[DeepAR Predicting]{DeepAR Predicting. 
    In the prediction steps, the target data \(y_{t-1}\) is not available. 
    To combat this, we sample \(\tilde{y}_{t-1} \sim \P(y_{t-1} | \boldsymbol{h}_{t-1})\) 
    from the previous step and use it as input instead of \(y_{t-1}\).}%
    \label{fig:deepar-predicting}%
\end{figure}

While the DeepAR model is trained by maximizing the likelihood function 
of the parametric distribution \(\P(y_t | \theta_t)\), 
the SQF-RNN model is trained by minimizing the CRPS scoring rule 
\(\mathrm{CRPS}(\P(\cdot | \theta_t), y_t)\) (see \ref{ch:crps}) 
which can be computed effectively for spline-based quantile functions (cf. \Textcite{Gasthaus2019} supplementary PDF).

A linear spline with \(L\) pieces is of the form 
\[ s(x; \gamma, b, d) = \gamma + \sum_{l=0}^L b_l (x - d_l)_+, 
\quad b, d \in \R^{L+1}. \]
Because the quantile function is a monotone function, 
a monotone spline is needed. That's why we need to create constraints 
for \(b_l\) and \(d_l\).
\(d_l < d_{l+1}\) results in ordered knot positions. To achieve this 
in the neural network, we set \(d_0 = 0\) and \(d_l = \sum_{j=1}^l \delta_j\), 
where \(\delta_j \geq 0\) and \(\sum_{j=1}^L \delta_j = 1\) since the domain 
of the quantile function is \([0, 1]\). 
The second property is monotonicity: the slope \(m_l\) between two knots is given by 
\(m_l = \sum_{j=0}^l b_j\). The constraint \(m_l \geq 0 \forall l\) solves that.
By setting \(b_l = \beta_l - \beta_{l-1}\) and \(b_0 = \beta_0\) with \(\beta_l \geq 0 \forall l\), 
\(m_l = \sum_{j=0}^l b_j = \beta_l \geq 0\) is ensured.
Therefore, the spline can be modeled with the parameter 
\(\theta = (\gamma, \beta, \delta)\), \(\gamma \in \R, \beta \in [0,\infty)^{L}, 
\delta \in \set{ \delta \in [0,1]^L: \sum_{j=1}^L \delta_j = 1 }\). 