\section{CRPS}
\label{ch:crps}

\renewcommand{\d}{\mathrm{d}}

The \gls{crps} is one of the most common 
scoring rules. It is defined as follows by \Textcite{Gneiting2014}: 

\[ \CRPS(F, y) = \int_{-\infty}^\infty \left( F(y) - \mathds{1} \{y \leq x\} \right)^2 \d x, \]
where \(F\) is the CDF of the probabilistic forecast.
We can show that the CRPS can also be written as 
\[ \CRPS(F, y) = 2 \int_0^1 L_\alpha(F^{-1}(\alpha), y) \d \alpha, \]
where \(L_\alpha\) is the pinball loss.
\begin{proof}
    The CRPS can be written as the integral over elementary scoring functions for quantile forecasts: 
    \[ \CRPS(F, y) = 2 \int_0^1 \int_{-\infty}^\infty \mathrm{s}_{\alpha, \eta}(F^{-1}(\alpha), y) \d \eta \d \alpha, \]
    where 
    \[ \mathrm{s}_{\alpha, \eta}(q, y) = \begin{cases}
        1-\alpha, &y\leq \eta < q, \\
        \alpha, &q\leq \eta < y, \\
        0, &\text{otherwise}.
    \end{cases} \]
    \Textcite{Ehm2016} state that elementary scoring functions are used to establish forecast 
    dominance over a class of scoring functions: if a forecast \(F_1\) has a lower score than 
    another forecast \(F_2\) for all elementary scoring functions of a certain class, 
    \(F_1\) dominates \(F_2\) with respect to the class \(\mathcal{P}\) 
    that is induced by these elementary scoring functions. We can check the dominance 
    with respect to the elementary scoring functions with so-called Murphy diagrams.

    Let \(\eta < y\). 
    \[ 2 \int_0^1 \mathrm{s}_{\alpha, \eta}(F^{-1}(\alpha), y) \d\alpha = 2 \int_0^{F(\eta)} \alpha \d \alpha = F(\eta)^2 = (F(\eta) - \mathds{1}_{\set{y \leq \eta}})^2. \]
    Let \(\eta \geq y\).
    \[ 2 \int_0^1 \mathrm{s}_{\alpha, \eta}(F^{-1}(\alpha), y) \d\alpha = 2 \int_{F(\eta)}^1 (1-\alpha) \d\alpha = \left[ - (1-\alpha)^2 \right]_{F(\eta)}^1 = (F(\eta) - \mathds{1}_{\set{y \leq \eta}})^2. \]
    Therefore, we get 
    \begin{align*}
        \CRPS(F, y) &= \int_{-\infty}^\infty (F(\eta) - \mathds{1}_{\set{y\leq x}})^2 \d \eta \\
        &= \int_{-\infty}^\infty 2 \int_0^1 \mathrm{s}_{\alpha, \eta}(F^{-1}(\alpha), y) \d \alpha \d \eta \\
        &= 2\int_0^1 \int_{-\infty}^\infty \mathrm{s}_{\alpha, \eta}(F^{-1}(\alpha), y) \d \eta \d \alpha  \tag{Tonelli}.
    \end{align*}
    Thus, we are finished after we conclude \(\int_{-\infty}^\infty \mathrm{s}_{\alpha, \eta}(F^{-1}(\alpha), y) \d \eta = L_\alpha(F^{-1}(\alpha), y)\).

    Let \(F^{-1}(\alpha) < y\). 
    \[ \int_{-\infty}^\infty \mathrm{s}_{\alpha, \eta}(F^{-1}(\alpha), y) \d \eta = \int_{F^{-1}(\alpha)}^y \alpha \d \eta = \alpha (y - F^{-1}(\alpha)) = L_\alpha(F^{-1}(\alpha), y). \]
    Let \(F^{-1}(\alpha) \geq y\). 
    \[ \int_{-\infty}^\infty \mathrm{s}_{\alpha, \eta}(F^{-1}(\alpha), y) \d \eta = \int_y^{F^{-1}(\alpha)} (1-\alpha) \d \eta = (1-\alpha) (F^{-1}(\alpha) - y) = L_\alpha(F^{-1}(\alpha), y). \]
\end{proof}