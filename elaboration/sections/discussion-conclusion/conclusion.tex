\section{Conclusion}
\label{sec:conclusion}

In this thesis, we compare three different approaches to probabilistic forecasting 
for solar energy generation and compare them on the GEFCom2014 dataset. 
Results show that the QRF and the NNQF model perform similar without 
hyperparameter optimization and that the QRF model improves after the hyperparameter 
optimization. The SQF-RNN model performs better during the summer months than the other two models. 

The QRF and NNQF model are both underdispersed during the night. This can be dealt with by simply forecasting \(0\). 
Nonparametric models are not always the easiest solution. Therefore, it makes sense to sometimes just predict the obvious 
instead of training a complex model.

The discrepancy between the results of the SQF-RNN model and the DeepAR model show that 
nonparametric models can perform better than parametric ones in a context where we do not know the actual probability distribution. 
The pinball loss crowned the QRF model the winner while the energy score marked the SQF-RNN model as winner. 
Different scoring functions emphasize different properties of a forecast -- just because one forecast has a 
better score than another forecast does not mean that it is better than the other in every aspect.

Another point to note is that different models perform differently due to different features being 
prioritized: the QRF and NNQF model both prioritize approximately the same features and therefore 
yield similar results. The SQF-RNN model prioritizes other features and performs in some months better 
and in other months worse than the other two models. A model's performance is highly dependent on the input features.
Therefore, it is important to investigate feature selection in detail and to examine which explaining features promise the best results. 
Future work could investigate feature selection and preprocessing in more detail. 

As discussed in section \ref{sec:discussion}, the SQF-RNN model does not perform as well as expected 
because the competition does not allow the previous time data to be used as an input value. 
Future work could examine how well the SQF-RNN and DeepAR model perform after fixing this issue, 
i.e., enabling \(\SI{24}{\hour}\) ahead forecasts with previous target data. 