\section{Spline Quantile Function RNNs}

Gasthaus et. al proposed a method for probabilistic forecasting by modeling 
the quantile function with monotonic regression splines. 
The proposed SQF-RNN model combines the ability to forecast time series 
from recurrent neural networks with the flexibility of being able to 
model the quantile functions with linear splines. 
The model is trained by minimizing the CRPS which can be computed effectively 
for spline-based quantile functions.

A linear spline with \(L\) pieces is of the form 
\[ s(x; \gamma, b, d) = \gamma + \sum_{l=0}^L b_l (x - d_l)_+, 
\quad b, d \in \R^{L+1}. \]
Since we want a monotone spline, we need to create constraints for \(b_l\) and \(d_l\).
We want \(d_l < d_{l+1}\) for ordered knot positions. To achieve this 
in the neural network, we set \(d_0 = 0\) and \(d_l = \sum_{j=1}^l \delta_j\), 
where \(\delta_j \geq 0\) and \(\sum_{j=1}^L \delta_j = 1\) since the domain 
of the quantile function is \([0, 1]\). 
We also want monotonicity: the slope \(m_l\) between two knots is given by 
\(m_l = \sum_{j=0}^l b_j\). We need to make sure that \(m_l \geq 0 \forall l\).
If we set \(b_l = \beta_l - \beta_{l-1}\) and \(b_0 = \beta_0\) with \(\beta_l \geq 0 \forall l\), 
we get \(m_l = \sum_{j=0}^l b_j = \beta_l \geq 0\).
We can therefore model our spline with the parameter 
\(\theta = (\gamma, \beta, \delta)\), \(\gamma \in \R, \beta \in [0,\infty)^{L}, 
\delta \in \set{ \delta \in [0,1]^L: \sum_{j=1}^L \delta_j = 1 }\).

Let \(z_{i,t} \in \R\) be the value of the \(i\)-th time series at time \(t\) and 
\(x_{i,t} \in \R^D\) be the corresponding predictors for the time series. 
The predictors are assumed to be available during prediction. In the 
GEFCom14 case, these are mainly data obtained from weather forecasting. 

The setup of the model is the following: 
During training, the model takes \(x_{i,t}\) as well as \(z_{i, t-1}\) 
and \(h_{i, t-1}\) -- the previous state of the network -- 
as input and calculates \(h_{i,t} = r(h_{i, t-1}, z_{i, t-1}, x_{i, t})\)
which is then used to calculate the quantile outputs 
\(\theta_{i,t} = \theta(h_{i,t}, \phi)\).
\(r(\cdot)\) is a multi-layer RNN with LSTM cells and \(\theta(\cdot)\) 
is a projection layer.
The quantiles are then used to calculate the CRPS loss and train the model.

Flunkert et al. (2017) proposed DeepAR which is also a 
probabilistic forecasting model with RNNs. It outputs the 
parameters of a probabilistic distribution as the forecast. 
The disadvantage of this method is that it requires the 
specification of the probabilistic distribution that fits the data. 
For complex problems, this is often not trivial. 
In the DeepAR implementation, the Student's \(t\)-distribution is used. 
This method performs noticably worse on the GEFCom14 dataset. 

The CRPS is often used to evaluate a forecast model but its usage as 
a direct loss function in the training process is rare. 
As the CRPS is closely related to the pinball loss 
(the CRPS is the integral over the pinball loss), this helps in 
the GEFCom14 problem since it directly minimizes the given metric.