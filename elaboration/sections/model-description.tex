\chapter{Model description}
\label{ch:model-description}

In this chapter, we give motivation why the models were selected and 
describe their basic functionality as well as their advantages.

Quantile Random Forests have proven to be a very competitive 
and powerful quantile regression method for high-dimensional datasets. 
This is the reason why we use it as a benchmark to compare the other two models 
against it.
Nearest Neighbor Quantile Regression is the method utilized by the first place 
in the GEFCom14 challenge. NNQF uses a similar strategy but requires much less 
computation power in order to predict its output. 
SQF-RNN uses autoregressive input with an RNN structure and provides flexible 
conditional output distributions through spline quantile functions. 
\Textcite{Gasthaus2019} already shows that it works well on large solar energy 
forecasting datasets.

In order to describe the model in the following subsections, we will introduce the necessary notation here:
\begin{itemize}
    \item \(n\) is the training data length
    \item \(x_1, \ldots, x_n \in \R^D\) are the predictor values of the time series
    \item \(y_1, \ldots, y_n \in \R\) are the corresponding target values
\end{itemize}

\section{Quantile Regression Forests}
\label{sec:qrf}

Quantile Regression Forests were first proposed by \Textcite{Meinshausen2006}
and have since then proven to be a powerful method for high-dimensional quantile 
regression and time series forecasting. 
The method works in a similar way as Random Forest with the main difference 
being that we don't take the mean over the trees but take different quantiles 
from the trees.

The performance of this algorithm is very competitive in comparison with other 
linear and tree-based methods.
Therefore, they provide a competitive baseline for this thesis. 

\Textcite{Meinshausen2006} also shows that Quantile Regression Forests are consistent 
under some specific assumptions about the distribution of the covariates, the proportion of observations, the splitting criterion and the 
continuity and monotonicity of the \gls{cdf}.

Quantile Regression Forests work as follows (cf. \Textcite{Meinshausen2006}):
\begin{enumerate}
    \item Grow \(k\) trees \(\func{T_1, \ldots, T_k}{\R^D}{\R}\) like in a Random Forest from the training data.
    \item For a given \(x\in\R^D\), calculate the outputs \( \tilde{y}_i = T_i(x) \) for all \(i \in \set{1, \ldots, k}\).
    \item Calculate the empirical quantiles \(y_{(0.01)}, \ldots, y_{(0.99)}\) from 
    \(\set{\tilde{y}_1, \ldots, \tilde{y}_k}\).
\end{enumerate}

\section{Nearest Neighbor Quantile Filters}
\label{sec:nnqf}

\Textcite{Ordiano2019} proposed a method for probabilistic 
energy forecasting using quantile regression based on a \gls{nnqf}. 
The method works as follows: first, the training set is modified 
by using the Nearest Neigbor Quantile Filters so that 
the training data directly represents a probabilistic distribution. 
Then, a regression model like an artificial neural network can 
train on this modified data set and learn the quantile function.

The preprocessing of the NNQF method includes multiple steps. 
Firstly, we search the \(k\) nearest neighbors 
for every \(x_i\).
The distance metric can be any distance metric on \(\R^D\), 
the euclidean metric is used often.
Let \(J \subset \set{1, \ldots, n}\) be the indices of 
those nearest neighbors. 
The probabilistic distribution of \(y_i\) can be approximated 
by calculating the empirical quantile \(\tilde{y}_{(q),i}\) of 
\(\set{y_j \;|\; j\in J}\) for each \(q \in \set{0.01, \ldots, 0.99}\). 

After repeating this procedure for each entry in the time series, 
we get vectors 
\[ \tilde{y}_{(q)} = \begin{pmatrix}
    \tilde{y}_{(q), 1} \\ 
    \vdots \\
    \tilde{y}_{(q), n}
\end{pmatrix} \]
that form the modified training set combined with the 
predictors \(X = (x_1, \ldots, x_n)\).

Because we work with a time series, adjacent points are correlated. 
That's the reason why we use lag features: 
instead of only \(x_i\), we are using \(x_i, \ldots, x_{i-H+1}\) to 
predict the target data. \(H\) is called lag size.

With the modified training set, one can now train the regression model 
for each quantile and fit the function 
\[ f_\theta(x_i, \ldots, x_{i-H+1}) = \tilde{y}_{(q), i}. \]
Common examples are polynomial regression or 
artificial neural networks. 

This method has three main advantages: 
\begin{enumerate}
    \item The technique for the quantile regression is not specified, 
    any technique can be used,
    \item the calculation of the nearest neighbors and the modified 
    training set only needs to be done once, you can save time when 
    using multiple quantile regressions. 
    \item The original dataset does not need to be saved afterwards, 
    we only need the weights of the regression model for predicting.
\end{enumerate}

In comparison to most other \(k\)-Nearest Neighbors quantile 
regression techniques, the nearest neighbors are only calculated once 
and then the regression model is trained on the modified training data. 
A regular \(k\)-Nearest Neighbor quantile regression algorithm 
calculates the nearest neighbors every time when a forecast is conducted 
(cf. \Textcite{Ma2015}, p. 3 ff) which is computationally more expensive in the long run.

\section{Spline Quantile Function RNNs}
\label{sec:sqf-rnn}

\Textcite{Gasthaus2019} proposed a method for probabilistic forecasting by modeling 
the quantile function with monotonic regression splines. 
The proposed \gls{sqfrnn} model combines the ability to forecast time series 
from recurrent neural networks which was already done by the DeepAR model from \Textcite{Salinas2017}
with the flexibility of being able to 
model the quantile functions with linear splines. 

Let \(x_1, \ldots, x_n \in \R^D\) be the predictor values and 
\(z_1, \ldots, z_n \in \R\) be the target time series. Also, let \(\Theta\) 
be the model parameters, \(\boldsymbol{h}_t\) the network output of 
time step \(t\) and \(\theta_t\) the parameters of the conditional distribution \(\P(z_t | \theta_t)\).
The model works as follows:
Compute the network output \(\boldsymbol{h}_t = h(\boldsymbol{h}_{t-1}, z_{t-1}, x_t, \Theta)\) 
as well as the parameters \(\theta_t = \theta(\boldsymbol{h}_t, \Theta)\) for the distribution
\(\P(z_t | \theta_t)\). \(h(\cdot)\) is a multi-layer RNN with 
LSTM cells and \(\theta(\cdot)\) is a projection layer. 
The quantiles are then used to calculate the loss and train the model parameters \(\Theta\).
The process is illustrated in Figure \ref{fig:deepar-training}.

\begin{figure}[h]%
    \centering
    \begin{tikzpicture}[yscale=-1,node distance=-\pgflinewidth]
    \tikzset{ReceptorNode/.style={circle, draw=black, fill=lightblue, thick, inner sep=2pt, minimum size=30pt}}
    \tikzset{Placeholder/.style={circle, thick, inner sep=2pt, minimum size=30pt}}
    \tikzset{Connection/.style={->, line width=0.5mm}}
    \newcommand{\mynode}[3]{
        \node[ReceptorNode] (circ-#2) at (#1, 0) {\(\boldsymbol{h}_{#2}\)};
        \node (x-#2) at (#1, 1.5) {\(x_{#2}, y_{#3}\)};
        \node (y-#2) at (#1, -1.5) {\(\P(y_{#2}|\boldsymbol{h}_{#2})\)};

        \draw[Connection] (circ-#2) -- (y-#2);
        \draw[Connection] (x-#2)    -- (circ-#2);
    }
    \newcommand{\placeholder}[2]{
        \node[Placeholder] (circ-#2) at (#1, 0) {\(\cdots\)};
        \node (x-#2) at (#1, 1.5) {};
        \node (y-#2) at (#1, -1.5) {\phantom{\(\P(y_{#2}|\boldsymbol{h}_{#2})\)}};
    }
    \newcommand{\connect}[2]{
        \draw[Connection] (circ-#1) -- (circ-#2);
    }

    % Create nodes
    \mynode{1 * 2.5}{1}{0}
    \onslide<2->{
        \mynode{2 * 2.5}{2}{1}
        \connect{1}{2}    
    }
    \onslide<2->{
        \mynode{3 * 2.5}{3}{2}
        \connect{2}{3}    
    }
    \onslide<3->{
        \placeholder{4 * 2.5}{4}
        \connect{3}{4}
    }
    % Last node is called "n"
    \onslide<4->{
        \mynode{5*2.5}{n}{n-1}
        \connect{4}{n}
    }
\end{tikzpicture}
    \caption{DeepAR Training}%
    \label{fig:deepar-training}%
\end{figure}

For the prediction step, the target time series are not known. 
The known history of the time series \(z_1, \ldots, z_{t_0}\) is fed into the 
model and for \(t > t_0\), samples \(\tilde{z}_t \sim \P(z_t | \theta_t)\) 
are generated and fed back into the model for the next time step.
The process is illustrated in Figure \ref{fig:deepar-predicting}.

\begin{figure}[h]%
    \centering
    \begin{tikzpicture}[yscale=-1,node distance=-\pgflinewidth]
    \tikzset{ReceptorNode/.style={circle, draw=black, fill=lightblue, thick, inner sep=2pt, minimum size=30pt}}
    \tikzset{Placeholder/.style={circle, thick, inner sep=2pt, minimum size=30pt}}
    \tikzset{Connection/.style={->, line width=0.5mm}}
    \newcommand{\mynode}[3]{
        \node[ReceptorNode] (circ-#2) at (#1, 0) {\(\boldsymbol{h}_{#2}\)};
        \node (x-#2) at (#1, 1.5) {\(x_{#2}, z_{#3}\)};
        \node (y-#2) at (#1, -1.5) {\(\P(z_{#2}|\boldsymbol{h}_{#2})\)};

        \draw[Connection] (circ-#2) -- (y-#2);
        \draw[Connection] (x-#2)    -- (circ-#2);
    }
    \newcommand{\mynodewithresult}[3]{
        \node[ReceptorNode] (circ-#2) at (#1, 0) {\(\boldsymbol{h}_{#2}\)};
        \node (x-#2) at (#1, 1.5) {\(x_{#2}, z_{#3}\)};
        \node (y-#2) at (#1, -1.5) {\(\P(z_{#2}|\boldsymbol{h}_{#2})\)};
        \node (z-#2) at (#1, -2.5) {\(\tilde{z}_{#2}\)};

        \draw[Connection] (circ-#2) -- (y-#2);
        \draw[Connection] (x-#2)    -- (circ-#2);
        \draw[Connection] (y-#2)    -- (z-#2);
    }
    \newcommand{\mynodewithresultinputsampled}[3]{
        \node[ReceptorNode] (circ-#2) at (#1, 0) {\(\boldsymbol{h}_{#2}\)};
        \node (x-#2) at (#1, 1.5) {\(x_{#2}, \tilde{z}_{#3}\)};
        \node (y-#2) at (#1, -1.5) {\(\P(z_{#2}|\boldsymbol{h}_{#2})\)};
        \node (z-#2) at (#1, -2.5) {\(\tilde{z}_{#2}\)};

        \draw[Connection] (circ-#2) -- (y-#2);
        \draw[Connection] (x-#2)    -- (circ-#2);
        \draw[Connection] (y-#2)    -- (z-#2);
    }
    \newcommand{\placeholder}[2]{
        \node[Placeholder] (circ-#2) at (#1, 0) {\(\cdots\)};
        \node (x-#2) at (#1, 1.5) {};
        \node[opacity=0] (y-#2) at (#1, -1.5) {\(\P(z_{#2}|h_{#2})\)};
    }
    \newcommand{\connect}[2]{
        \draw[Connection] (y-#1)    -- (circ-#2);
        \draw[Connection] (circ-#1) -- (circ-#2);
    }

    % Create nodes
    \mynode{1 * 2.5}{T}{T-1}
    \draw[Connection] ([xshift=-0.5cm]circ-T.west) -- (circ-T);
    \onslide<2->{
        \mynodewithresult{2 * 2.5}{T+1}{T}
        \connect{T}{T+1}
    }
    \onslide<3->{
        \mynodewithresultinputsampled{3 * 2.5}{T+2}{T+1}
        \connect{T+1}{T+2}
    }
    \onslide<4->{
        \placeholder{4 * 2.5}{T+3}
        \connect{T+2}{T+3}
    }
    % Last node is called "T+n"
    \onslide<5->{
        \mynodewithresultinputsampled{5*2.5}{T+n}{T+n-1}
        \connect{T+3}{T+n}
    }
\end{tikzpicture}
    \caption{DeepAR Predicting}%
    \label{fig:deepar-predicting}%
\end{figure}

While the DeepAR model is trained by maximizing the likelihood function, 
the SQF-RNN model is trained by minimizing the CRPS (see \ref{ch:crps}) 
which can be computed effectively for spline-based quantile functions.

A linear spline with \(L\) pieces is of the form 
\[ s(x; \gamma, b, d) = \gamma + \sum_{l=0}^L b_l (x - d_l)_+, 
\quad b, d \in \R^{L+1}. \]
Since we want a monotone spline, we need to create constraints for \(b_l\) and \(d_l\).
We want \(d_l < d_{l+1}\) for ordered knot positions. To achieve this 
in the neural network, we set \(d_0 = 0\) and \(d_l = \sum_{j=1}^l \delta_j\), 
where \(\delta_j \geq 0\) and \(\sum_{j=1}^L \delta_j = 1\) since the domain 
of the quantile function is \([0, 1]\). 
We also want monotonicity: the slope \(m_l\) between two knots is given by 
\(m_l = \sum_{j=0}^l b_j\). We need to make sure that \(m_l \geq 0 \forall l\).
If we set \(b_l = \beta_l - \beta_{l-1}\) and \(b_0 = \beta_0\) with \(\beta_l \geq 0 \forall l\), 
we get \(m_l = \sum_{j=0}^l b_j = \beta_l \geq 0\).
We can therefore model our spline with the parameter 
\(\theta = (\gamma, \beta, \delta)\), \(\gamma \in \R, \beta \in [0,\infty)^{L}, 
\delta \in \set{ \delta \in [0,1]^L: \sum_{j=1}^L \delta_j = 1 }\).