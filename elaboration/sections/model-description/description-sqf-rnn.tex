\section{Spline Quantile Function RNNs}
\label{sec:sqf-rnn}

\Textcite{Gasthaus2019} proposed a method for probabilistic forecasting by modeling 
the quantile function with monotonic regression splines. 
The proposed \gls{sqfrnn} model combines the ability to forecast time series 
from recurrent neural networks with the flexibility of being able to 
model the quantile functions with linear splines. 
The model is trained by minimizing the CRPS (\ref{ch:crps}) which can be computed effectively 
for spline-based quantile functions.

A linear spline with \(L\) pieces is of the form 
\[ s(x; \gamma, b, d) = \gamma + \sum_{l=0}^L b_l (x - d_l)_+, 
\quad b, d \in \R^{L+1}. \]
Since we want a monotone spline, we need to create constraints for \(b_l\) and \(d_l\).
We want \(d_l < d_{l+1}\) for ordered knot positions. To achieve this 
in the neural network, we set \(d_0 = 0\) and \(d_l = \sum_{j=1}^l \delta_j\), 
where \(\delta_j \geq 0\) and \(\sum_{j=1}^L \delta_j = 1\) since the domain 
of the quantile function is \([0, 1]\). 
We also want monotonicity: the slope \(m_l\) between two knots is given by 
\(m_l = \sum_{j=0}^l b_j\). We need to make sure that \(m_l \geq 0 \forall l\).
If we set \(b_l = \beta_l - \beta_{l-1}\) and \(b_0 = \beta_0\) with \(\beta_l \geq 0 \forall l\), 
we get \(m_l = \sum_{j=0}^l b_j = \beta_l \geq 0\).
We can therefore model our spline with the parameter 
\(\theta = (\gamma, \beta, \delta)\), \(\gamma \in \R, \beta \in [0,\infty)^{L}, 
\delta \in \set{ \delta \in [0,1]^L: \sum_{j=1}^L \delta_j = 1 }\).

Let \(z_{i,t} \in \R\) be the value of the \(i\)-th time series at time \(t\) and 
\(x_{i,t} \in \R^D\) be the corresponding predictors for the time series. 
The predictors are assumed to be available during prediction. 

The setup of the model is the following: 
During training, the model takes \(x_{i,t}\) as well as \(z_{i, t-1}\) 
and \(h_{i, t-1}\) -- the previous state of the network -- 
as input and calculates \(h_{i,t} = r(h_{i, t-1}, z_{i, t-1}, x_{i, t})\)
which is then used to calculate the quantile outputs 
\(\theta_{i,t} = \theta(h_{i,t}, \phi)\).
\(r(\cdot)\) is a multi-layer RNN with LSTM cells and \(\theta(\cdot)\) 
is a projection layer.
The quantiles are then used to calculate the CRPS loss and train the model.