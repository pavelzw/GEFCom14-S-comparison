\section{Motivation}
\label{sec:motivation}

Energy forecasting has become increasingly important in the last years 
because of higher energy consumption and a higher demand for renewable energies. 
Solar energy power generation can be characterized through weather variables like 
precipitation, solar radiation and cloud cover. 
That's why \Textcite{Hong2016} hosted a forecasting competition in 2014. 

\Textcite{Hong2016} state that wind power forecasting is 
more mature than solar energy forecasting. This is because 
wind power forecasing is the closest to meteorological forecasting, 
where probabilistic forecasting is well-established and commonly accepted.
Although solar power forecasting has the same subdomain as wind forecasting, 
namely renewable energy generation, it is neither in the 
point forecasting nor in the probabilistic forecasting domain as mature 
as wind power forecasting. 

Because modeling the uncertainty of a forecast is relevant for planning 
the energy household of the electric grid, probabilistic forecasts 
make sense instead of only point forecasts. 

This is the reason why we will be comparing three different nonparametric 
machine learning models to tackle the probabilistic solar energy forecasting 
problem. 
The three models that we take a look at are 
\begin{enumerate}
    \item Quantile Regression Forests proposed by \Textcite{Meinshausen2006},
    \item Nearest Neighbor Quantile Filters proposed by \Textcite{Ordiano2019},
    \item Spline Quantile Functions RNN proposed by \Textcite{Gasthaus2019}.
\end{enumerate}
All of these models are so called distributional regression methods, meaning 
they predict a probabilistic distribution and try to minimize a loss function 
given the distribution and the target data.