\section{Related Work}
\label{sec:related-work}

This section covers related work concerning solar energy forecasting as well as 
the different models that we look at. 
\Textcite{Hong2016} discusses the importance of solar energy, wind energy, electric load 
and electric price forecasting. They summarize the recent research progress on probabilistic 
energy forecasting and introduce the Global Energy Forecasting Competition 2014 (GEFCom2014).
\Textcite{Meinshausen2006} proposes the Quantile Random Forests method. He describes 
how it works, gives numerical examples that undermine its predictive performance
and provides proofs for some of the model's properties like consistency. 
\Textcite{Ordiano2019} introduce the Nearest Neighbors Quantile Filters (NNQF) method. 
They explain the advantages and test the model's performance on the GEFCom2014 dataset. 
We will also try out the model on the GEFCom2014 dataset and evaluate 
it in multiple kinds of manners.
\Textcite{Salinas2017} introduce the DeepAR model -- a model for probabilistic forecasting 
with autoregressive recurrent neural networks. They evaluate the model on several 
real-world data sets and show that this model has accuracy improvements of around 
\(15\%\) compared to other state-of-the-art methods.
\Textcite{Gasthaus2019} propose the SQF-RNN model, an extension of the DeepAR model 
that is supposed to be a more flexible method for probabilistic forecasting 
by modeling the quantile function with splines instead of a fixed distribution class.
They empirically demonstrate its effectiveness by evaluating the model on several real-world 
data sets and comparing it with the DeepAR model.