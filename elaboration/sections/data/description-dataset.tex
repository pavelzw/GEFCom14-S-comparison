\section{The GEFCom2014 dataset}

In order to compare different energy forecasting methods, 
Tao Hong et. al organized the Global Energy Forecasting Competition 2014 (GEFCom2014), 
a probabilistic energy forecasting competition with four tracks: 
electric load, electric prica, wind power and solar power forecasting. 
The competition attracted 581 participants from 61 countries. 

The Global Energy Forecasting Competition took place the first time in 2012. 
In the 2014 competition, they upgraded the competition with three features: 
\begin{enumerate}
    \item instead of point forecasts, probabilistic forecasts were used;
    \item four forecasting tracks were used: electric load (L track), 
    electric price (P track), wind power (W track) and solar power (S track);
    electric load an wind forecasting were also used in 2012, solar power and 
    electric price forecasting were added;
    \item incremental data releases on a weekly basis to mimic real world forecasting.
\end{enumerate}

In this thesis, we will only look at the solar power track. 
In this track, you need to predict the power generation of three 
solar power plants (the so called zones) in Australia. The prediction should be on a rolling 
basis for 24 h ahead. 
For the forecasts, you get different prediction variables shown in Table 1. 
%todo create table!

\subsection{Error measure}

To measure the error of a probabilistic forecast, one usually uses a 
proper scoring rule. 
Let \(\mathcal{P}\) be a convex class of probability measures on the 
measurable space \((\Omega, \mathcal{A})\).
A \textit{scoring rule} is any extended real-valued function 
\[ \func{S}{\mathcal{P} \times \Omega}{\overline{\R}} \]
such that \(S(P, \cdot)\) is \(\mathcal{P}\)-quasiintegrable for all 
\(P\in \mathcal{P}\).
A scoring rule \(S\) is \textit{proper} relative to a convex subclass 
\(\mathcal{P}_0 \subseteq \mathcal{P}\) if
\[ \int_\Omega S(Q, \omega) \mathrm{d}Q(\omega) \leq \int_\Omega S(P, \omega) \mathrm{d}Q(\omega) \]
for all \(P, Q \in \mathcal{P}_0\). It is \textit{strictly proper} 
if equality holds iff. \(P = Q\).

This property encourages honest and careful quotes by the forecaster. 
To evaluate the different forecast competitors in GEFCom2014, 
the pinnball loss was used -- mainly because of the ease of implementation 
and because the pinnball loss is a proper scoring rule.
For each time period over the forecast horizon, the participants need to 
provide the quantiles \(q_{0.01}, \ldots, q_{0.99}\). 
\(q_0 = -\infty\) and \(q_1 = \infty\) are the natural lower and upper bounds.
The pinball loss \(L\) is defined as: 
\[ L(q_p, y) = \begin{cases}
    (1-p)(q_p - y), &\text{if } y < q_p, \\
    p(y - q_p), &\text{if } y \geq q_p,
\end{cases} \]
where \(p \in \set{0.01, \ldots, 0.99}\) and \(y\) is the observed target.

The overall score is the average over all time points, zones and quantiles.
The models are then compared with a benchmark which is a basic point forecast 
that was created using a na\"{i}ve model.