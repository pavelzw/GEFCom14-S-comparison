\pdfbookmark[2]{Zusammenfassung}{Zusammenfassung}

\chapter*{Zusammenfassung}

\begin{center}
  \begin{minipage}{12cm}
    \begin{sloppypar}
      Mit der steigenden Bedeutung von regenerativen Energiequellen ergibt es immer mehr Sinn, 
      sich mit der Vorhersage von regenerativer Energieerzeugung zu beschäftigen. 
      Um die Unsicherheit in Vorhersagen zu modellieren, sagt man häufig die Wahrscheinlichkeitsverteilung der 
      Zufallsvariable vorher. Es gibt einige verschiedene nichtparametrische Machine Learning Herangehensweisen 
      für das Solarenergieerzeugungsproblem. Deshalb vergleicht diese Arbeit drei verschiedene Möglichkeiten, Solarenergie 
      vorherzusagen, nämlich Quantile Random Forests, Nearest Neighbor Quantile Filters sowie Spline Quantile Functions RNN. 
      Die Leistung der Modelle werden durch deren Leistung im Global Energy Forecasting Competition 2014 Datensatz verglichen. 
      Wir sehen, dass Quantile Regression Forests den besten Score von \(0{,}01873\) erzielen. Nearest Neighbor Quantile Filters 
      erreicht einen Score von \(0{,}0194\) und Spline Quantile Functions RNN erreichen einen Score von \(0{,}02041\).
    \end{sloppypar}
  \end{minipage}
\end{center}