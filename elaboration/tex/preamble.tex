% remove indentation throughout the document
\setlength{\parindent}{0pt}

\usepackage{tocbasic}

%%% Doc: ftp://tug.ctan.org/pub/tex-archive/macros/latex/contrib/caption/caption.pdf
\usepackage[tableposition=above]{caption}

% Aussehen der Captions
\captionsetup{
   margin = 10pt,
   font = {rm},
   labelfont = {bf},
   format = plain, % oder 'hang'
   indention = 0em,  % Einruecken der Beschriftung
   labelsep = space, %period, space, quad, newline
   justification = RaggedRight, % justified, centering, RaggedRight
   singlelinecheck = true, % false (true=bei einer Zeile immer zentrieren)
   position = bottom %top
}

%%% Bugfix Workaround
\DeclareCaptionOption{parskip}[]{}
\DeclareCaptionOption{parindent}[]{}

\usepackage{csquotes}
\usepackage{siunitx}

%% useful abreviations

\newcommand\ie{i.\,e.\xspace}
\newcommand\eg{e.\,g.\xspace}
\newcommand\Eg{E.\,g.\xspace}
\newcommand\NB{N.\,B.\xspace}
\newcommand\BSc{B.\,Sc.\xspace}
\newcommand\MSc{M.\,Sc.\xspace}
\newcommand\PhD{Ph.\,D.\xspace}
\newcommand\etc{etc.\xspace}
\newcommand\cf{cf.\xspace}
\newcommand\Cf{Cf.\xspace}
\newcommand\etal{et\,al.\xspace}
\newcommand\page[1]{p.\,#1}
\newcommand\pages[1]{pp.\,#1}
\newcommand\ham{a.\,m.\xspace}
\newcommand\hpm{p.\,m.\xspace}
	
\newcommand\zB{z.\,B.\xspace}
\newcommand\proz{\,\%\xspace}

%% Useful definitions for tables --------------------------------------------------------- %% 

% um Tabellenspalten mit Flattersatz zu setzen, muss \\ vor
% (z.B.) \raggedright geschuetzt werden:
\newcommand{\PreserveBackslash}[1]{\let\temp=\\#1\let\\=\temp}

% Linksbuendig:
\newcolumntype{v}[1]{>{\PreserveBackslash\RaggedRight\hspace{0pt}}p{#1}}
\newcolumntype{M}[1]{>{\PreserveBackslash\RaggedRight\hspace{0pt}}m{#1}}
\newcolumntype{Y}{>{\PreserveBackslash\RaggedLeft\hspace{0pt}}X}
%%% Spalten fuer Mathematik
%
% serifenlose Matheschrift
%\newcolumntype{s}[1]{%
%  >{\DC@{.}{,}{#1}\mathsf\bgroup}l%
%  <{\egroup\DV@end}%
%}

% Tabellenspaltentyp fuer den Kopf: (Farbe + Ausrichtung)
\newcolumntype{H}[1]{>{\columncolor{tableheadcolor}}l}

%%% ---|Layout der Tabellen |-------------------

% Neue Umgebung fuer Tabellen:

\newenvironment{Tabelle}[2][c]{%
  \tablestylecommon
  \begin{longtable}[#1]{#2}
  }
  {\end{longtable}%
  \tablerestoresettings
}

% Groesse der Schrift in Tabellen
\newcommand{\tablefontsize}{ \footnotesize}
\newcommand{\tableheadfontsize}{\footnotesize}

% Layout der Tabelle: Ausrichtung, Schrift, Zeilenabstand
\newcommand\tablestylecommon{%
  \renewcommand{\arraystretch}{1.4} % Groessere Abstaende zwischen Zeilen
  \normalfont\normalsize            %
  \sffamily\tablefontsize           % Serifenlose und kleine Schrift
  \centering%                       % Tabelle zentrieren
}

\newcommand{\tablestyle}{
  \tablestylecommon
  %\tablealtcolored
}

% Ruecksetzten der Aenderungen
\newcommand\tablerestoresettings{%
  \renewcommand{\arraystretch}{1}% Abstaende wieder zuruecksetzen
  \normalsize\rmfamily % Schrift wieder zuruecksetzen
}

% Tabellenkopf: Serifenlos+fett+schraeg+Schriftfarbe
\newcommand\tableheads{%
  \tableheadfontsize%
  \sffamily\bfseries%
  %\slshape
  %\color{white}
}

\newcommand\tablesubheadfont{%
  \tableheadfontsize%
  \sffamily\bfseries%
  \slshape
  %\color{white}
}

\newcommand\tableheadcolor{%
  %\rowcolor{tablesubheadcolor}
  %\rowcolor{tableblackheadcolor}
  \rowcolor{tableheadcolor}%
}

\newcommand\tablesubheadcolor{%
  \rowcolor{tablesubheadcolor}
  %\rowcolor{tableblackheadcolor}
}

\newcommand{\tableend}{\arrayrulecolor{black}\hline}

% Tabellenkopf (1=Spaltentyp, 2=Text)
% \newcommand{\tablehead}[2]{
%   \multicolumn{1}{#1@{}}{%
%     \raisebox{.1mm}{% Ausrichtung der Beschriftung
%       #2%
%     }\rule{0pt}{4mm}}% unsichtbare Linie, die die Kopfzeile hoeher macht
% }


\newcommand{\tablesubhead}[2]{%
  \multicolumn{#1}{>{\columncolor{tablesubheadcolor}}l}{\tablesubheadfont #2}%
}

% Tabellenbody (=Inhalt)
\newcommand\tablebody{%
\tablefontsize\sffamily\upshape%
}

\newcommand\tableheadshaded{%
  \rowcolor{tableheadcolor}%
}
\newcommand\tablealtcolored{%
  \rowcolors{1}{tablerowcolor}{white!100}%
}
%%% --------------------------------------------

\usepackage{ragged2e}

%% Rotate table head
%% http://tex.stackexchange.com/questions/98388/how-to-make-table-with-rotated-table-headers-in-latex
%% http://tex.stackexchange.com/questions/32683/rotated-column-titles-in-tabular
\newcommand*\rottblhead{\rotatebox{75}}

\usepackage{cleveref}

\newcommand\stress[1]{{\color{red}#1}} 

% \mathds{1}
\usepackage{dsfont}

\usepackage{tabularx}

\usepackage{siunitx}